\subsection*{Thesis outline}

In this section we will attempt to draw a rough outline of what the thesis will look like.
The thesis will consist of the following sections:

\begin{itemize}
    \item Introduction
    \begin{itemize}
        \item What is the OPF problem?
        \item Why is it important?
        \item Further math details(some of the implementation details might belong in Methodology)
        \begin{itemize}
            \item Optimization problem statement
            \item (Maybe) more physics explanations for CS people
            \item How is the grid modelled?
            Buses, generators, lines, admittance matrix, limits etc.
            \item What are the constants and variables in the problem?
            \item What are the node masks and why do they have the values that they have?
            \item More\ldots
        \end{itemize}
        \item How is it solved conventionally and why that isn't good enough?
        \item How can ML methods help?
        \item Intro into (classical) ML methods for Power people.
        \item Why can't classical ML methods handle multi-topology datasets?
        \item What are graphs, graph neural networks and why are they well suited to fill the gap?
        \item Why can't GNNs learn long range dependencies and why we speculate that those are crucial for solving the OPF problem?
        \item What are graph transformers?
        \begin{itemize}
            \item What are transformers?
            \item Vanilla transformers VS GNNs - Pros and Cons(expressivity, complexity)
            \item What are positional and structural encodings?
        \end{itemize}
        \item How can the problem be approached?(supervised/unsupervised)
        \item More\ldots
    \end{itemize}
    \item Literature overview
    \begin{itemize}
        \item Which datasets do current works use and why are they inconsistent?
        \item What metrics do they use and why are they not good enough?
        \item Which models do they use?
        \item What are the inputs and outputs of the models?
    \end{itemize}
    \item Methodology
    \begin{itemize}
        \item Datasets
        \begin{itemize}
            \item On which datasets are the models going to be trained on?
            \item Description of the datasets(what kind of graphs, some graph stats, voltage level, single-/multi-topology, number of nodes, other info etc.
            \item Data info - shapes and sizes, maybe $\pqva$ statistics.
        \end{itemize}
        \item Metrics
        \begin{itemize}
            \item What metrics am I going to use and why?
            \item Units, pros, cons, aggregation, other info
        \end{itemize}
        \item Models
        \begin{itemize}
            \item Model pipeline (stuff that I have in the Appendix now)
            \begin{itemize}
                \item Pipeline diagram
                \item Why do we standardize inputs and outputs?
            \end{itemize}
            \item On each model (starting from Linear$_{local}$ and increasing the complexity incrementally)
            \begin{itemize}
                \item Diagram
                \item What are the inputs and outputs?
                \item What's the math behind it?
                \item What's the motivation behind applying it
                \item Hyperparameters and implementation details (maybe here, maybe in the Appendix)
            \end{itemize}
        \end{itemize}
    \end{itemize}
    \item Discussion
    \begin{itemize}
        \item What results did all the trained models achieve?
        \item Which are good, which are bad?
        \item What were the difficulties?
        \item What do the results mean?
        \item What is a satisfying result?
        \item Are the results satisfying?
    \end{itemize}
    \item Conclusion
\end{itemize}

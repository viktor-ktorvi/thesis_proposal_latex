The upper bound error $\uppererror{X} \in \mathbb{R}_{\ge 0}$ and
the lower bound error $\lowererror{X} \in \mathbb{R}_{\le 0}$ defined
in table~\ref{tab:inequalityconstraintmetrics} represent the signed
magnitude of the respective bound violation if a violation exists, otherwise,
if no violation exists, the value of the metrics is zero.
Neither of these metrics are scale invariant for the same reason mentioned in the
section on equality constraints, so it would be worth thinking about adding
a relative version of these metrics.
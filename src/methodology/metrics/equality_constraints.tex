The absolute error $\absoluteerror{X}$, defined in table~\ref{tab:equalityconstraintmetrics},
is simply the absolute difference between the power at the given node and
the in- and out-flowing powers.
The difference follows from the power flow equations embedded in the equality
constraints defined in the system of equations~\ref{eq:opf_formulation}:

\begin{equation}
    \label{eq:active_power_error}
    \Delta \boldsymbol{P} = \boldsymbol{P} - \Re\left\{ \complexpower \right\}
\end{equation}

\begin{equation}
    \label{eq:reactive_power_error}
    \Delta \boldsymbol{Q} = \boldsymbol{Q} - \Im\left\{ \complexpower \right\}
\end{equation}

The main issue with this metric is that it's not scale invariant i.e.\
it's not the same if the same value of the metric is achieved on a
node representing a power plant and one representing a single house.
It is for that reason that we introduce the relative absolute error
$\relativeabsoluteerror{X}$, also defined in table~\ref{tab:equalityconstraintmetrics},
which takes the absolute error at the given node and divides it with
the absolute value of the power of the node, thus making it scale
invariant.
The relative absolute error still isn't perfect because it's
overly sensitive to nodes with lower powers, for example,
if the power at a node is really close to zero, even a
negligible absolute error would blow up to produce a high
value of the relative error, so, while analyzing, it makes sense to mainly
look at $\relativeabsoluteerror{X}$ while keeping an eye on
$\absoluteerror{X}$ as a sanity check, as well as being aware of the
orders of magnitude of the powers in the grid.
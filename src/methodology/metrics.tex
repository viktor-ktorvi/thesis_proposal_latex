\subsection*{Metrics}
To improve the interpretability of the metrics used to assess the quality of the achieved OPF solution
we introduce metrics with regard to: equality constraints, inequality constraints and objective function.
The metrics categories are given in descending order with respect to importance.
Equality constraints are the most important because the solution must obey the laws of physics;
then come the inequality constraints representing the technical (and regulatory) limitations
which must be fulfilled for the proper functioning of the power grid;
finally, the objective function value comes last in priority because only when the
first two constraints are fulfilled can we start thinking about minimizing costs.

\subsubsection*{Equality constraints}
From the problem formulation~\ref{eq:opf_formulation} we can derive the active and reactive power errors:

\begin{equation}
    \label{eq:active_power_error}
    \Delta \boldsymbol{P} = \boldsymbol{P} - \Re\left\{ \complexpower \right\}
\end{equation}

\begin{equation}
    \label{eq:reactive_power_error}
    \Delta \boldsymbol{Q} = \boldsymbol{Q} - \Im\left\{ \complexpower \right\}
\end{equation}

\noindent It will be useful to look at absolute values of these errors in the per unit system and, additionally, we'd like to get an aggregate value
for each on the entire validation/test dataset, so we will look at the average and maximum value
per node since we don't want the metric to depend on the number of nodes in a grid.
We define the mean and maximum absolute power error as:

\newcommand{\abs}[1]{\left| #1 \right|}

\newcommand{\expectedvalue}[1]{\mathbb{E} \left\{#1\right\}}

\newcommand{\expectedabsvalue}[1]{\expectedvalue{\abs{#1}}}

\newcommand{\expectedrelativeabsvalue}[1]{\expectedabsvalue{\frac{\Delta #1}{#1}}}


\begin{equation}
    \label{eq:mean_absolute_power_error}
    \expectedabsvalue{\Delta X} = \frac{1}{M} \sum_i^{M} \abs{\Delta X_i}
\end{equation}

\begin{equation}
    \label{eq:max_absolute_power_error}
    \max \ \abs{\Delta X} = \max_i{\abs{\Delta X_i}} \quad i \in \{0, M - 1\}
\end{equation}

\noindent where $X \in \{P, Q\}$ and $M$ is the number of nodes in a dataset. \\

Metrics~\ref{eq:mean_absolute_power_error} and~\ref{eq:max_absolute_power_error} still
have the issue that their values are dependent on the scale of the powers in a grid, for example,
when looking at two nodes, one representing a household and the other a power plant, the same
value of the absolute power error can represent a high constraint violation to the household
node but a negligible violation to the power plant.
It is for this reason that we introduce the mean and maximum relative absolute power error representing
the ratio between the error and value at the node, thus getting rid of the problem of scale:

\begin{equation}
    \label{eq:mean_relative_absolute_power_error}
    \expectedrelativeabsvalue{X} = \frac{1}{M} \sum_i^{M} \abs{\frac{\Delta X_i}{X_i}}
\end{equation}

\begin{equation}
    \label{eq:max_relative_absolute_power_error}
    \max \ \abs{\frac{\Delta X}{X}} = \max_i{\abs{\frac{\Delta X_i}{X_i}}} \quad i \in \{0, M - 1\}
\end{equation}

\subsubsection*{Inequality constraints}
\subsubsection*{Objective function}
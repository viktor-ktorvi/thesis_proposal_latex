\usepackage{lstmisc}To analyze the current state of applying graph neural networks to
OPF, in Table~\ref{tab:resultstableref} we provide the results of training three models on a single-topology dataset, each having access to
wider neighbourhood information.
The details of how the models were implemented, trained and selected can be found in the~\nameref{chap:appendix}.
Intuitively, to solve the conservation of energy equations, one would need information on the
state of all the nodes in the grid.
Doing this for single-topology datasets is simple as we only need to concatenate the
features of each node into a vector and fit, for instance, a linear model to the dataset.
And this does, as seen in Table~\ref{tab:resultstableref},
solve the problem with a satisfactory degree of accuracy.
However, if we want to train models on a multi-topology dataset, which is a real world use case,
this approach would not work as the classical machine learning algorithms cannot work with
this data due to the varying number of features in the graph.
It is for this reason that we have to resort to graph learning approaches.
In the same table, we see the effects of applying:

\begin{itemize}
    \item Linear$_{local}$ - a linear model with no access to neighbourhood features.
    \item GCN - a graph convolutional network~\cite{kipf2016semi} which incorporates the features of the
    local neighbourhood of each node.
    \item GCN-JK: a GCN with jumping knowledge~\cite{xu2018representation} which enables the GCN to have a bit
    more depth and thus reach more nodes.
    \item Linear$_{global}$ - a linear model which incorporates features of every node in the grid
    by concatenating them.
\end{itemize}

\noindent We can see that, as expected, incorporating more and more neighbourhood information leads to
smaller values of $\relativeabsoluteerror{P}$ and $\relativeabsoluteerror{Q}$.


In practice, grid operators need to solve the OPF problem for thousands of variations of the node parameters and even for variations of the
underlying topology of the grid.
For example, beginning with house owners or office buildings wanting to install heat pumps or a rooftop photovoltaic array,
to larger entities installing wind and solar farms, each of these events constitutes a change of the underlying graph structure for
which a large number of OPF calculations need to be run to ensure that the stability of the grid and technical constraints are maintained.
Today, these calculations are performed with solvers, like the interior point method, whose run time is slow and optimality of
the solution isn't guaranteed\cite{cain2012history}.
Machine learning methods can't improve the optimality guarantees, but they can improve the run time by orders of magnitude by learning a heuristic that approximately solves the OPF problem.
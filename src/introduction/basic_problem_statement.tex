The optimal power flow problem (OPF) is a nonlinear, non-convex optimization problem\cite{cain2012history, bienstock2019strong} whose algebraic formulation is dictated by a graph,
specifically the graph representation of a power grid.
The most basic formulation of the problem consists of three distinct parts:

\begin{itemize}
    \item the objective function - an economic cost function of the active powers produced by the generators in
    the grid and the corresponding costs of running those generators.

    \item the equality constraints - representing the physical law of conservation of energy.

    \item the inequality constraints - representing the technical limitations of running the grid which dictate that
    the voltages and powers of specific nodes(buses) must be within some predefined ranges.
\end{itemize}

The problem boils down to minimizing the operating cost while fulfilling the physical and technical constraints.
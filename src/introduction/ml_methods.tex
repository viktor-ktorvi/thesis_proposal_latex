Machine learning methods for solving OPF aim to learn a heuristic which produces
a solution of sufficient quality while being much faster than conventional solvers.
These methods could be trained in one of two ways - supervised and unsupervised.
In both cases a power system operator would generate a dataset of power grids with
supply tasks representative of their intended use case.
This may include variations on the generation and demand profiles in both normal
and extreme grid conditions as well as variations of the grid topology representing
new users being connected to the grid, line expansion or disconnections from the
grid.
In the supervised learning case, a conventional solver would be run for each sample
in the dataset, necessitating a high upfront computational cost and a machine
learning model would then be trained on a regression task.
On the other hand, in an unsupervised setting, a machine learning model could be
trained to minimize, for example, the lagrangian of the OPF problem thereby,
if successful, circumventing the high upfront computational cost of solving the
dataset which is a more desirable outcome.
Of course, both methods still require a high upfront computational cost of
training, but given enough usage of the model during its lifetime, that cost
would be outweighed by the time saved on not having to use conventional solvers.
Graph neural networks(GNNs) naturally lend themselves to the OPF problem since it
lives on a graph, because they can incorporate graph structure and line attributes and can
handle datasets with multiple topologies, whereas multilayer perceptrons(MLPs)
can only be applied to single-topology datasets.

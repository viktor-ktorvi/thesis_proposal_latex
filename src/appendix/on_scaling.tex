One minor trick that leads to better performance and that can only be applied to single-topology datasets is computing the statistics
$\mu_{in}, \sigma_{in} \in \mathbb{R}^{N_{const}}$ and $\mu_{out}, \sigma_{out} \in \mathbb{R}^{N_{var}}$.
This is contrary to the more general case applicable to multi-topology datasets where we compute
$\mu_{in}, \sigma_{in}, \mu_{out}, \sigma_{out} \in \mathbb{R}^4$ only over the physical values themselves.
Additionally, in this case we calculate the input statistics after applying the node mask to the $\pqva$ matrix
(essentially deleting the target info) because that is what the input layer of the network will be receiving, but
use the untouched $\pqva$ matrix for calculating the output statistics because there is no reason for the zeros
to skew the statistics.

\mbox{}\\
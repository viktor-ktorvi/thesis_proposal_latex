There exists a decent amount of literature on applying machine learning methods to the OPF problem.
Because of their ability to generalize to multiple grid topologies, in this overview we'll focus on
methods that utilize GNNs. Of those works, however, most suffer from one or multiple reoccurring problems -
incomparable dataset use, lack of interpretable metrics and inconsistent definitions of input and output
variables of the problem.

There exists a noticeable lack of a benchmark dataset and for that reason the authors
of each work generate their own dataset by taking some common example grid topology
and randomly augmenting the grid
parameters~\cite{liu2022topology, liu2021graph, yang2022ac, owerko2020optimal, owerko2022unsupervised}
or by modelling the inclusion of renewable energy sources into those same grid
topologies~\cite{surani2023graph, gao2023physics}.
The fact that every work uses a different dataset makes results difficult to compare.
It's worth mentioning that generating datasets based on models of the power grid is likely a legitimate approach, given the lack of accessible real world power grid data.
Power grid datasets, if they exist, are often kept private by the power system operators.
Additionally, the amount of metering in distribution grids is very low so most of the time even
system operators don't have access to the real world data.
Another noticeable shortcoming of most works is a lack of interpretable metrics of
the quality of the OPF solution which would be very important if you were to convince operators
to start using such methods.
Often what is presented are just the MSE or MAE used to train
the networks~\cite{gao2023physics, liu2022topology, liu2021graph}, leaving a lot of room to improve the metrics.
Thirdly, even though some works~\cite{owerko2020optimal, owerko2022unsupervised} take
in the entire state of the grid(the matrix $[\boldsymbol{P} \ \boldsymbol{Q} \ \boldsymbol{V} \ \boldsymbol{\theta}]$),
none of them try to predict the entire state of the grid, thus leaving some information unknown.
Instead, some works predict only the node level optimal costs~\cite{surani2023graph, liu2021graph},
others just the active powers and/or voltage magnitudes.
Finally, all of these works use 1-WL bound GNNs leaving room to research the application of more expressive methods.